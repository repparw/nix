\documentclass[11pt,a4paper,sans]{moderncv}
\moderncvstyle{casual}
\moderncvcolor{blue}

\usepackage[utf8]{inputenc}
\usepackage[T1]{fontenc}
\usepackage{lmodern}
\usepackage[scale=0.85]{geometry}

\setlength{\footskip}{20pt}
\setlength{\hintscolumnwidth}{2.5cm}

\name{Ulises}{Britos}
\title{Desarrollador de Software}
\address{La Plata}{Buenos Aires, Argentina}
\phone[mobile]{+54~3329~333263}
\email{ubritos@gmail.com}
\social[github]{repparw}

\begin{document}
\makecvtitle

\section{Perfil Profesional}
Desarrollador de software orientado a resultados, con experiencia en desarrollo web y administración de sistemas. Destaco por la resolución de problemas y el trabajo en equipo. Comprometido con el aprendizaje continuo y la implementación de soluciones eficientes utilizando tecnologías modernas.

\section{Habilidades Técnicas}
\cvitem{Lenguajes}{Python, Java, JavaScript, Ruby, Nix}
\cvitem{Desarrollo}{React.js, Node.js, HTML5, CSS3, APIs RESTful}
\cvitem{DevOps}{Git, Docker, NixOS, Linux, Administración de Sistemas}
\cvitem{Bases de Datos}{PostgreSQL, SQLite, SQL}
\cvitem{Metodologías}{Agile, Scrum, Control de Versiones}

\section{Formación Académica}
\cventry{2019--Presente}{Licenciatura en Informática}{Universidad Nacional de La Plata}{}{}{
Materias relevantes: Estructuras de Datos, Algoritmos, Ingeniería de Software, Bases de Datos}

\cventry{2012--2018}{Bachiller - Orientación Contabilidad}{EES N°1}{Moquehuá}{}{}

\section{Proyectos}
\cventry{2023--Presente}{Configuración NixOS y Homelab Personal}{}{}{}{
\begin{itemize}
\item Implementación y mantenimiento de configuración NixOS declarativa para múltiples sistemas
\item Gestión de servicios auto-hospedados: Jellyfin (multimedia), FreshRSS (noticias), Paperless (documentos)
\item Implementación de autenticación centralizada con Authelia y gestión segura de secretos
\item Automatización de despliegues y control de versiones mediante Git, CI/CD y containerización
\end{itemize}}

\section{Idiomas}
\cvitem{Español}{Nativo}
\cvitem{Inglés}{Nivel Avanzado (Profesional)}

\end{document}
